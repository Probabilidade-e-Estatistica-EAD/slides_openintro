%%%%%%%%%%%%%%%%%%%%%%%%%%%%%%%%%%%%%%%%%%%%%%%%%%%%%%%%%%%%%%

\section{1.1.Estudo de Caso}



%%%%%%%%%%%%%%%%%%%%%%%%%%%%%%%%%%%%

\begin{frame}
\frametitle{Tratando Síndrome de Fadiga Crônica}


\begin{itemize}
\justifying
\item \hl{Objetivo:} Avaliar a eficácia da terapia cognitiva-comportamental na síndrome da fadiga crônica.


\item \hl{Amostra:}  médicos da atenção primária e consultores de uma clínica especializada em síndrome da fadiga crônica encaminharam 142 pacientes.


\item \hl{Amostragem:} Apenas 60 dos 142 pacientes encaminhados entraram no estudo. Alguns foram excluídos porque não preenchiam os critérios de diagnóstico, alguns tinham outros problemas de saúde e outros simplesmente se recusaram a participar do estudo.

\end{itemize}

\justifying
\ct{Deale et. al. \textit{Terapia comportamental cognitiva para síndrome da fadiga crônica: um estudo randomizado controlado}. The American Journal of Psychiatry 154.3 (1997).}

\end{frame}

%%%%%%%%%%%%%%%%%%%%%%%%%%%%%%%%%%%%

\begin{frame}
\frametitle{Delineamento do estudo}

\begin{itemize}
\justifying
\item Os pacientes foram aleatoriamente alocados para os grupos controle e tratamento, 30 pacientes em cada grupo:
\begin{itemize}
\justifying
\item \hl{Tratamento}: Terapia comportamental cognitiva - colaborativa e educativa. Foi explicado aos pacientes que a atividade física pode aumentar de forma constante e segura sem piorar os sintomas da doença.
\justifying
\item \hl{Controle:} Relaxamento - Nenhum conselho foi dado sobre como a atividade física poderia ser desenvolvida. Em vez disso, foram ensinados métodos de relaxamento muscular progressivo.
\end{itemize}

\end{itemize}

\end{frame}

%%%%%%%%%%%%%%%%%%%%%%%%%%%%%%%%%%%%

\begin{frame}
\frametitle{Resultados}
\justifying
A tabela abaixo mostra a distribuição de bom ou mal resultado para os pacientes aos 6 meses de estudo. Note que 7 pacientes abandonaram o estudo: 3 do grupo tratamento e 4 do grupo controle.

\begin{center}
\begin{tabular}{ll  cc c} 
			&				& \multicolumn{2}{c}{\textit{Bom resultado}} \\
\cline{3-4}
			&							& Sim 	& Não 	& Total	\\
\cline{2-5}
							&Tratamento 	& 19	 	& 8		& 27 	\\
\raisebox{1.5ex}[0pt]{\textit{Grupo}}	&Controle		& 5	 	& 21	 	& 26 \\
\cline{2-5}
							&Total		& 24		& 29		& 53
\end{tabular}
\end{center}

\end{frame}
%%%%%%%%%%%%%%%%%%%%%%%%%%%%%%%%%%%%

\begin{frame}
\frametitle{Resultados}
\begin{itemize}

\item Proporção de pacientes com bons resultados no grupo tratamento:
\[ 19/27 \approx 0.70 \rightarrow 70\% \]

\pause

\item Proporção de pacientes com bons resultados no grupo controle:
\[ 5 / 26 \approx 0.19 \rightarrow 19\% \]

\end{itemize}

\end{frame}

%%%%%%%%%%%%%%%%%%%%%%%%%%%%%%%%%%%%

\begin{frame}
\frametitle{Compreendendo os resultados}

\dq{Os dados mostram uma diferença "real" entre os grupos?}

\pause

\begin{itemize}
\justifying
\item Suponha que você jogue uma moeda 100 vezes. Embora a chance de observar cada um dos lados da moeda seja de 50\%, provavelmente, não observaremos exatamente 50 caras e 50 coroas. Esse tipo de flutuação faz parte da maioria dos processos geradores de dados.

\justifying
\item A diferença observada entre os dois grupos (70 - 19 = 51 \%) pode ser real, ou pode ser devido à variação natural.

\justifying
\item Como a diferença é muito grande, é crível que realmente exista diferença entre os grupos.
\end{itemize}

\end{frame}


\begin{frame}
\frametitle{Compreendendo os resultados}
\begin{itemize}


\justifying
\item Precisamos de ferramentas estatísticas para determinar se a diferença é tão grande que devemos rejeitar a noção de que foi devido ao acaso.

No caso do estudo da fadiga crônica, a diferença na proporção de bons resultados do grupo tratamento e do grupo controle é devido a flutuações aleatórias ou é um indício de que o tratamento cognitivo é um protocolo mais eficiente?

\end{itemize}

\end{frame}

%%%%%%%%%%%%%%%%%%%%%%%%%%%%%%%%%%%%

\begin{frame}
\frametitle{Generalizando os resultados}

\justifying
\dq{Os resultados deste estudo são generalizáveis para todos os pacientes com síndrome da fadiga crônica?}

\pause

\justifying
Esses pacientes tinham características específicas e se voluntariaram para fazer parte deste estudo, portanto, podem não ser representativos de todos os pacientes com síndrome da fadiga crônica. Embora não possamos generalizar imediatamente os resultados para todos os pacientes, este primeiro estudo é encorajador. O método funciona para pacientes com um conjunto restrito de características e isso dá esperança de que funcionará, pelo menos em algum grau, com outros pacientes.


\end{frame}

%%%%%%%%%%%%%%%%%%%%%%%%%%%%%%%%%%%%

